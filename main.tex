\documentclass[12pt]{article}
\usepackage[utf8]{inputenc}
\usepackage{amsmath}
\usepackage{hyperref}
\usepackage{graphicx}
\title{Klossmetoden}
\date{}
\begin{document}
  \maketitle
  
  \section{Introduktion}
  Ett problem med Skolvekets kursplaner (egentligen \textit{ämnesplan}) för gymnasiematten är att de lämnar stort tolkningsutrymme både i centralt innehåll och i kunskapskraven.
  Ett annat problem är att det är svårt att se hur kunskapskraven hänger samman med det centrala innehållet, vilket gör det svårt att veta hur man bedömer elever som exempelvis hanterar statistik och geometri men inte algebra och ekvationer.

  Klossmetoden är tänkt att fungera som ett lager utanpå kursplanerna, och göra det lättare att förstå dem.
  Målet med metoden är dels att ge lärare stöd i planerings- och bedömningsarbete, dels att göra det lättare att diskutera och nå samstämmighet i frågor om kursinnehåll och kunskapskrav.

  Givetvis är det fortfarande styrdokumenten som gäller ytterst, och Klossmetoden försöker bara komplettera dem – inte på något sätt ersätta dem.

  Metoden är fortfarande under utveckling.
  Om du använder den får du gärna berätta om dina erfarenheter till \href{mailto:johan.falk@rudbeck.se}{Johan Falk}, lärare på Rudbeck i Sollentuna, på mail johan@vaxjonexus.com.
  Du kan även diskutera den här metoden (och andra saker) på \href{https://www.facebook.com/groups/matematikundervisning/}{facebook.com/groups/matematikundervisning}.
 
  \section{Annat experimentellt}
  % This is a comment; it is not shown in the final output.
  % The following shows a little of the typesetting power of LaTeX
  \begin{align}
    E &= mc^2                              \\
    m &= \frac{m_0}{\sqrt{1-\frac{v^2}{c^2}}}
  \end{align}
  
  External figure:
  \externalfigure{http://a.rgbimg.com/cache1nHn7t/users/s/su/sundstrom/300/mifyTKG.jpg}
  
  write18:
  \write18{http://a.rgbimg.com/cache1nHn7t/users/s/su/sundstrom/300/mifyTKG.jpg}
  \includegraphics{mifyTKG.jpg}
  
  \begin{table*}
\begin{tabular}

  \subsection{Lite mer experimentellt}
  
\end{tabular}
\caption{En sorts tabell}
\label{En sorts etikett}
  å & ä & ö \\
  a & b & c \\
\end{table*}
  
\begin{figure}
\centering
\end{figure}  
\end{document}
